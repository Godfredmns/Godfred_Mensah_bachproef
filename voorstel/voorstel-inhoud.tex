%---------- Inleiding ---------------------------------------------------------

% TODO: Is dit voorstel gebaseerd op een paper van Research Methods die je
% vorig jaar hebt ingediend? Heb je daarbij eventueel samengewerkt met een
% andere student?
% Zo ja, haal dan de tekst hieronder uit commentaar en pas aan.

%\paragraph{Opmerking}

% Dit voorstel is gebaseerd op het onderzoeksvoorstel dat werd geschreven in het
% kader van het vak Research Methods dat ik (vorig/dit) academiejaar heb
% uitgewerkt (met medesturent VOORNAAM NAAM als mede-auteur).
% 

\section{Inleiding}%
\label{sec:inleiding}

In de afgelopen jaren is het aantal data dat organisaties beheren in hedendaagse bedrijfsapplicaties aanzienlijk gestegen, waardoor de nood aan een doordacht en gestructureerd datakwaliteitsbeheer steeds sterker naar voren komt. Binnen veel ondernemingen speelt SAP Master Data Governance een centrale rol in het vastleggen en controleren van masterdata over verschillende domeinen heen, maar het tijdig detecteren van fouten en inconsistenties blijft een uitdaging doordat traditionele validatiemechanismen vaak gebaseerd zijn op statische regels en beperkte context. Ook binnen Alluvion, waar SAP MDG wordt ingezet in verschillende projecten, komt deze uitdaging duidelijk naar voren. Binnen het vakgebied groeit daarom de interesse in artificiële intelligentie, dat in staat is patronen en afwijkingen te identificeren die met klassieke validatieregels moeilijk te herkennen zijn. Daarom onderzoekt dit voorstel hoe AI praktisch kan worden ingezet om de datakwaliteit binnen SAP MDG te ondersteunen

\subsection{Probleemstelling}
Hoewel SAP Master Data Governance (SAP MDG) organisaties ondersteunt bij het centraal beheren en controleren van masterdata, blijft het in de praktijk moeilijk om de kwaliteit van deze gegevens op een consistente manier te bewaken. In projecten waar Alluvion SAP MDG inzet, blijkt dat veel data manueel wordt ingevoerd door verschillende gebruikers, waardoor velden soms onvolledig, onnauwkeurig of inconsistent zijn. Daarnaast komen gegevens vaak uit meerdere externe bronnen, wat leidt tot verschillen in structuur en kwaliteit. De bestaande validatieregels, die meestal gebaseerd zijn op vaste BRF+ regels, kunnen alleen fouten herkennen die vooraf expliciet zijn vastgelegd. Hierdoor blijven subtiele fouten of nieuwe foutpatronen onopgemerkt, wat extra controles en aanpassingen vereist. Deze uitdagingen tonen aan dat er nood is aan een flexibelere manier om fouten sneller en nauwkeuriger te identificeren dan momenteel mogelijk is met de beschikbare validatiemiddelen.
\subsection{Hoofdonderzoeksvraag}
Dit onderzoek vertrekt vanuit de volgende centrale onderzoeksvraag:
\begin{quote}
    \textbf{Hoe kan artificiële intelligentie bijdragen tot een hogere datakwaliteit en automatische foutdetectie binnen SAP Master Data Governance?}
\end{quote}
\subsection{Deelvragen}

Om deze centrale onderzoeksvraag te beantwoorden, worden zowel deelvragen geformuleerd binnen het \textbf{probleemdomein} als binnen het \textbf{oplossingsdomein}. Deze helpen om het probleem grondig te analyseren en de gepaste oplossingsrichtingen te onderzoeken.

\subsubsection{Deelvragen binnen het probleemdomein}

\begin{enumerate}
    \item Welke soorten fouten, inconsistenties en anomalieën komen het vaakst voor in master data binnen SAP MDG?
    \item Hoe gebeurt validatie van master data momenteel binnen SAP MDG en welke beperkingen brengen de huidige validatiemechanismen met zich mee?
    \item Welke impact hebben foutieve of onvolledige master data op bedrijfsprocessen?
\end{enumerate}

\subsubsection{Deelvragen binnen het oplossingsdomein}

\begin{enumerate}
    \item Welke AI-technieken zijn geschikt voor foutdetectie en anomaliedetectie in gestructureerde masterdata?
    \item Hoe kan een AI-model worden getraind en geëvalueerd op basis van historische of synthetische data?
    \item In welke mate kunnen AI-modellen bijdragen tot automatische dataverrijking met behulp van externe bronnen?
    \item Hoe verhoudt de performantie van een AI-model zich tot klassieke SAP-validatieregels op het vlak van nauwkeurigheid en betrouwbaarheid?
\end{enumerate}

\subsection{Onderzoeksdoelstellingen}
Het doel van dit onderzoek is te bepalen in welke mate AI-technieken een meerwaarde kunnen bieden ten opzichte van traditionele validatiebenaderingen. Concreet worden verschillende modellen onderzocht die fouten, inconsistenties en anomalieën in vendor master data automatisch kunnen identificeren en, waar mogelijk, gegevens kunnen verrijken met behulp van externe bronnen.

Deze bachelorproef is geslaagd wanneer het ontwikkelde AI-model effectief foutdetectie kan uitvoeren met voldoende nauwkeurigheid, stabiliteit en betrouwbaarheid, en wanneer overtuigend kan worden aangetoond dat deze aanpak een meetbare meerwaarde biedt ten opzichte van bestaande validatieregels.


%---------- Stand van zaken ---------------------------------------------------

\section{Literatuurstudie}%
\label{sec:literatuurstudie}


% Voor literatuurverwijzingen zijn er twee belangrijke commando's:
% \autocite{KEY} => (Auteur, jaartal) Gebruik dit als de naam van de auteur
%   geen onderdeel is van de zin.
% \textcite{KEY} => Auteur (jaartal)  Gebruik dit als de auteursnaam wel een
%   functie heeft in de zin (bv. ``Uit onderzoek door Doll & Hill (1954) bleek
%   ...'')

Uit studies blijkt dat organisaties vaak problemen hebben met fouten in masterdata. Deze fouten ontstaan door manuele invoer, verschillende databronnen of onduidelijke processen. Traditionele validatie werkt meestal met vaste regels en kan daardoor alleen fouten vinden die vooraf zijn bepaald \autocite{rahm2000data}. Onderzoek toont dat zulke regels weinig flexibel zijn en moeilijk omgaan met nieuwe of onverwachte fouten \autocite{ebraheem2018distributed}. Andere studies geven aan dat klassieke controles vooral eenvoudige fouten vinden, maar minder geschikt zijn voor inhoudelijke of complexe inconsistenties \autocite{batini2009methodologies}.

In de literatuur wordt artificiële intelligentie voorgesteld als een manier om fouten beter op te sporen. AI kan patronen en afwijkingen herkennen zonder dat hiervoor vooraf regels moeten worden gemaakt \autocite{chandola2009anomaly}. Verschillende studies tonen dat AI-technieken beter zijn in het vinden van subtiele of ongebruikelijke fouten dan klassieke validatiesystemen \autocite{aggarwal2015outlier}. Ook blijkt dat deep-learningmodellen fouten in gegevens in tabelvorm kunnen ontdekken die met gewone regels moeilijk zichtbaar zijn \autocite{hulsebos2019sherlock}.

%---------- Methodologie ------------------------------------------------------
\section{Methodologie}%
\label{sec:methodologie}

Het onderzoek wordt uitgevoerd in verschillende fasen die samen moeten aantonen of artificiële intelligentie kan helpen bij het verbeteren van datakwaliteit binnen SAP MDG.

\begin{itemize}
    \item \textbf{Fase 1: Analyse van het huidige validatieproces}
    \begin{itemize}
        \item In kaart brengen welke fouten vaak voorkomen in masterdata.
        \item Onderzoeken hoe SAP MDG en BRF+ deze fouten vandaag controleren.
        \item De beperkingen van het huidige validatiekader identificeren.
    \end{itemize}

    \item \textbf{Fase 2: Opstellen en voorbereiden van de dataset}
    \begin{itemize}
        \item Samenstellen van een geschikte dataset (synthetisch, geanonimiseerd of beide).
        \item Data opschonen en voorbereiden zodat AI-modellen ermee kunnen werken.
        \item Indien nodig fouten labelen voor evaluatie.
    \end{itemize}

    \item \textbf{Fase 3: Het verkennen en kiezen van AI-technieken}
    \begin{itemize}
        \item Onderzoeken welke AI-methoden geschikt zijn voor foutdetectie.
        \item Selecteren van één of meerdere modellen om te testen.
    \end{itemize}

    \item \textbf{Fase 4: Trainen en evalueren van het AI-model}
    \begin{itemize}
        \item Het gekozen model trainen op de voorbereide dataset.
        \item Prestaties meten met metrics zoals nauwkeurigheid, precisie, recall en F1-score.
        \item Resultaten vergelijken met traditionele SAP-validatieregels.
    \end{itemize}

    \item \textbf{Fase 5: Uitwerken van een proof-of-concept}
    \begin{itemize}
        \item Een demonstratie maken die laat zien hoe het AI-model fouten opspoort.
        \item Nagaan of de aanpak praktisch toepasbaar is binnen een SAP-omgeving.
    \end{itemize}

    \item \textbf{Fase 6: Analyse en conclusie}
    \begin{itemize}
        \item De resultaten samenvatten en interpreteren.
        \item Vaststellen in welke situaties AI een meerwaarde biedt.
        \item Beperkingen en mogelijke verbeterpunten bespreken.
    \end{itemize}
\end{itemize}


%---------- Verwachte resultaten ----------------------------------------------
\section{Verwacht resultaat, conclusie}%
\label{sec:verwachte_resultaten}

Het verwachte eindresultaat van deze bachelorproef is een AI-model dat fouten in masterdata beter kan vinden dan de huidige validatieregels in SAP MDG. Het model zou verschillende soorten fouten moeten kunnen herkennen, ook wanneer deze niet vooraf in regels zijn vastgelegd. Daarom wordt verwacht dat de AI-aanpak hogere scores haalt op metingen zoals precisie, recall en F1-score. Dit kan leiden tot minder manuele controles, efficiëntere dataverwerking en betrouwbaardere gegevens voor verdere bedrijfsprocessen. De aanpak is niet beperkt tot één soort masterdata en kan in verschillende delen van SAP MDG worden toegepast.

Op basis van de metingen en vergelijkingen zal het onderzoek duidelijk maken hoeveel meerwaarde AI biedt voor foutdetectie. Als de AI-modellen beter presteren dan de bestaande validatieregels, laat dit zien dat AI een nuttige aanvulling kan zijn op het huidige datakwaliteitsproces. Als de resultaten minder sterk zijn dan verwacht, biedt dit inzicht in de beperkingen van deze aanpak en welke voorwaarden nodig zijn om AI succesvol toe te passen. In beide gevallen levert dit onderzoek waardevolle inzichten op voor organisaties die hun datakwaliteit willen verbeteren en vormt het een basis voor toekomstige uitbreidingen.